\documentclass[UTF8]{ctexrep}

\usepackage{fancyhdr}
\usepackage[letterpaper, left=1in, right=1in, top=1in, bottom=1in]{geometry}
\usepackage{sectsty}
\usepackage{graphicx}
\usepackage{subfig}
\usepackage[section]{placeins}
\usepackage{hyperref}
\usepackage{amsmath}
\usepackage{listings}
\usepackage{color}
\usepackage{lstautogobble}

\definecolor{dkgreen}{rgb}{0,0.6,0}
\definecolor{gray}{rgb}{0.5,0.5,0.5}
\definecolor{mauve}{rgb}{0.58,0,0.82}

\lstset{frame=tb,
    language            = C,
    aboveskip           = 3mm,
    belowskip           = 5mm,
    showstringspaces    = false,
    columns             = flexible,
    basicstyle          = {\small\ttfamily},
    numbers             = none,
    numberstyle         = \tiny\color{gray},
    keywordstyle        = \color{blue},
    commentstyle        = \color{dkgreen},
    stringstyle         = \color{mauve},
    breaklines          = true,
    breakatwhitespace   = true,
    tabsize             = 3,
    numbers             = left,
    numberblanklines    = true,
    firstnumber         = 1,
    numberstyle         = \scriptsize\color{black},
    numbersep           = 12pt,
    escapeinside        = ||,
    mathescape          = true,
    autogobble          = true
}

\usepackage{xcolor}

\definecolor{codegreen}{rgb}{0,0.6,0}
\definecolor{codegray}{rgb}{0.5,0.5,0.5}
\definecolor{codepurple}{rgb}{0.58,0,0.82}
\definecolor{backcolour}{rgb}{0.95,0.95,0.92}

\lstdefinestyle{mystyle}{
    backgroundcolor=\color{backcolour},
    commentstyle=\color{codegreen},
    keywordstyle=\color{magenta},
    numberstyle=\tiny\color{codegray},
    stringstyle=\color{codepurple},
    basicstyle=\ttfamily\footnotesize,
    breaklines=true,
    captionpos=b,
    keepspaces=true,
    showspaces=false,
    showstringspaces=false,
    showtabs=false,
    tabsize=4
}

\lstset{style=mystyle}
\let\origthelstnumber\thelstnumber
\makeatletter
\newcommand*\Suppressnumber{%
  \lst@AddToHook{OnNewLine}{%
    \let\thelstnumber\relax%
     \advance\c@lstnumber-\@ne\relax%
    }%
}

\newcommand*\Reactivatenumber[1]{%
  \setcounter{lstnumber}{\numexpr#1-1\relax}
  \lst@AddToHook{OnNewLine}{%
   \let\thelstnumber\origthelstnumber%
   \refstepcounter{lstnumber}
  }%
}


\makeatother
\hypersetup{
    colorlinks=true,
    linkcolor=blue,
    filecolor=magenta,
    urlcolor=cyan,
}
\allsectionsfont{\mdseries\scshape}

\renewcommand{\thesection}{\arabic{section}}

\newcommand{\horrule}[1]{\rule{\linewidth}{#1}}
\title{
    \horrule{0.5pt} \\[0.4cm]
    \huge 2021年第八届中国可视化与可视分析大会\\
    数据可视分析挑战赛\\
    (ChinaVis Data Challenge 2021)\\
    作品说明文档\\
    \horrule{2pt}
}
\author{
    方志成 \ 黄霖 \ 陈雪韬 \ 陈思贝
}
\date{
    % TODO: Date
    2021.6
}
\setcounter{section}{-1}

\begin{document}
    \maketitle

    \section{参赛信息}

    \begin{itemize}
        \item 参赛队名称:
        \item 作品名称:
        \item 作品主题关键词:
        \item 团队成员:
        \begin{enumerate}
            \item 方志成,上海交通大学,fangzhicheng@stju.edu.cn,队长
            \item 黄霖,上海交通大学,@stju.edu.cn
            \item 陈雪韬,上海交通大学,@stju.edu.cn
            \item 陈思贝,上海交通大学,tonychen21@stju.edu.cn
            \item 董笑菊,上海交通大学,@stju.edu.cn,指导老师
        \end{enumerate}
        \item 团队成员是否与报名表一致:是
        \item 是否学生队:是
        \item 使用的分析工具或开发工具:Express, d3, 高德开放平台
        \item 共计耗费时间(人天):60
        \item 本次比赛结束后,我们是否可以在网络上公布该文档与相关视频:
    \end{itemize}

    \section{作品简介}
    % 请围绕作品主题、要解决的问题\场景、目标用户\读者、应用价值等方面简要介绍作品(建议参赛者描述本部分内容不多于500字,图表不多于1个)
    % TODO: 黄霖

    \section{数据介绍}
    % 请围绕数据来源、数据格式、数据严谨性、数据清洗等方面简要介绍(建议参赛者描述本部分内容不多于500字,图表不多于3个)
    % TODO: 陈思贝

    \section{分析任务与可视分析总体流程}
    % (建议参赛者描述本部分内容不多于500字,图表不多于3个)
    % TODO: 陈雪韬

    \section{数据处理与算法模型}
    % (建议参赛者描述本部分内容不多于1000字,图表不多于5个)
    % TODO: 

    \section{可视化与交互设计}
    % (建议参赛者描述本部分内容不多于1500字,图表不多于5个)

    \subsection{热力图}
    

    \subsection{风向图}

    \subsection{污染源}

    \subsection{均值图}

    \section{实验 \textbackslash 案例 \textbackslash 场景分析}
    % (建议参赛者描述本部分内容不多于2000字,图表不多于10个)

    \subsection{大气污染源分析}
    % 利用可视分析技术,识别主要大气污染源,分析关键污染成因。(可以根据自身情况联合其他数据辅助分析)
    % TODO: 陈思贝

    \subsection{大气污染时空态势分析}
    % 利用可视分析技术,分析大气污染时空分布模式、监控大气污染时空演变态势。
    % TODO: 

    \subsection{大气污染传输模式分析}
    % 利用可视分析技术,比较各地大气污染物差异、大气污染传输模式、检测异常传输事件,制定传输防治策略。
    % TODO: 陈雪韬

    \subsection{大气污染预测}
    % 利用可视分析技术,预测大气污染发展趋势、重污染天气预警。
    % TODO:

    \subsection{大气环境的改善}
    % 利用可视分析技术,展示大气污染治理过程中的大气环境状况、评估大气污染防治措施。
    % TODO:

    \section{讨论与总结}
    % (建议参赛者描述本部分内容不多于500字)
    % TODO: 方志成
    

\end{document}